%%%%%%%%%%%%%%%%%%%%%%%%%%%%%%%%%%%%%%%%%%%%%%%%%%%%%%%%%%%%%%%%%%%%%%%%
%%%%%%%%%%%%%%%%%%%%%% Simple LaTeX CV Template %%%%%%%%%%%%%%%%%%%%%%%%
%%%%%%%%%%%%%%%%%%%%%%%%%%%%%%%%%%%%%%%%%%%%%%%%%%%%%%%%%%%%%%%%%%%%%%%%

%%%%%%%%%%%%%%%%%%%%%%%%%%%%%%%%%%%%%%%%%%%%%%%%%%%%%%%%%%%%%%%%%%%%%%%%
%% NOTE: If you find that it says                                     %%
%%                                                                    %%
%%                           1 of ??                                  %%
%%                                                                    %%
%% at the bottom of your first page, this means that the AUX file     %%
%% was not available when you ran LaTeX on this source. Simply RERUN  %%
%% LaTeX to get the ``??'' replaced with the number of the last page  %%
%% of the document. The AUX file will be generated on the first run   %%
%% of LaTeX and used on the second run to fill in all of the          %%
%% references.                                                        %%
%%%%%%%%%%%%%%%%%%%%%%%%%%%%%%%%%%%%%%%%%%%%%%%%%%%%%%%%%%%%%%%%%%%%%%%%

%%%%%%%%%%%%%%%%%%%%%%%%%%%% Document Setup %%%%%%%%%%%%%%%%%%%%%%%%%%%%

% Don't like 10pt? Try 11pt or 12pt
\documentclass[1pt]{article}
\usepackage{times}
\usepackage[T1]{fontenc}
\usepackage{lmodern}
\renewcommand*\familydefault{\sfdefault} %% Only if the base font of the document is to be sans serif

% This is a helpful package that puts math inside length specifications
\usepackage{%
	calc,%
	%enumitem,%
	hyperref%
}
\usepackage{paralist}

% Layout: Puts the section titles on left side of page
\reversemarginpar

%
%         PAPER SIZE, PAGE NUMBER, AND DOCUMENT LAYOUT NOTES:
%
% The next \usepackage line changes the layout for CV style section
% headings as marginal notes. It also sets up the paper size as either
% letter or A4. By default, letter was used. If A4 paper is desired,
% comment out the letterpaper lines and uncomment the a4paper lines.
%
% As you can see, the margin widths and section title widths can be
% easily adjusted.
%
% ALSO: Notice that the includefoot option can be commented OUT in order
% to put the PAGE NUMBER *IN* the bottom margin. This will make the
% effective text area larger.
%
% IF YOU WISH TO REMOVE THE ``of LASTPAGE'' next to each page number,
% see the note about the +LP and -LP lines below. Comment out the +LP
% and uncomment the -LP.
%
% IF YOU WISH TO REMOVE PAGE NUMBERS, be sure that the includefoot line
% is uncommented and ALSO uncomment the \pagestyle{empty} a few lines
% below.
%

%% Use these lines for letter-sized paper
%\usepackage[paper=letterpaper,
            %%includefoot, % Uncomment to put page number above margin
            %marginparwidth=1.2in,     % Length of section titles
            %marginparsep=.05in,       % Space between titles and text
            %margin=1in,               % 1 inch margins
            %includemp]{geometry}

%% Use these lines for A4-sized paper
%\usepackage[paper=a4paper,
%            includefoot, % Uncomment to put page number above margin
%            marginparwidth=30.5mm,    % Length of section titles
%            marginparsep=1.5mm,       % Space between titles and text
%            margin=25mm,              % 25mm margins
%            includemp]{geometry}

%% Use these lines for maximum space utilization for CV or resume
\usepackage[paper=a4paper,
            includefoot, % Uncomment to put page number above margin
            marginparwidth=27.5mm,    % Length of section titles
            marginparsep=0mm,       % Space between titles and text
            margin=16mm,              % 16mm margins
            includemp]{geometry}

%% More layout: Get rid of indenting throughout entire document
\setlength{\parindent}{0in}

%% This gives us fun enumeration environments. compactenum will be nice.
\usepackage{paralist}

%% Reference the last page in the page number
%
% NOTE: comment the +LP line and uncomment the -LP line to have page
%       numbers without the ``of ##'' last page reference)
%
% NOTE: uncomment the \pagestyle{empty} line to get rid of all page
%       numbers (make sure includefoot is commented out above)
%
\usepackage{fancyhdr,lastpage}

\pagestyle{fancy}
\pagestyle{empty}      % Uncomment this to get rid of page numbers
\fancyhf{}\renewcommand{\headrulewidth}{0pt}
\fancyfootoffset{\marginparsep+\marginparwidth}
\newlength{\footpageshift}
\setlength{\footpageshift}
          {0.5\textwidth+0.5\marginparsep+0.5\marginparwidth-2in}
\lfoot{\hspace{\footpageshift}%
       \parbox{4in}{\, \hfill %
                    \arabic{page} of \protect\pageref*{LastPage} % +LP
%                    \arabic{page}                               % -LP
                    \hfill \,}}

% Finally, give us PDF bookmarks
\usepackage{color,hyperref}
\definecolor{darkblue}{rgb}{0.0,0.0,0.3}
\hypersetup{colorlinks,breaklinks,
            linkcolor=darkblue,urlcolor=darkblue,
            anchorcolor=darkblue,citecolor=darkblue}

%%%%%%%%%%%%%%%%%%%%%%%% End Document Setup %%%%%%%%%%%%%%%%%%%%%%%%%%%%


%%%%%%%%%%%%%%%%%%%%%%%%%%% Helper Commands %%%%%%%%%%%%%%%%%%%%%%%%%%%%

% The title (name) with a horizontal rule under it
%
% Usage: \makeheading{name}
%
% Place at top of document. It should be the first thing.
\newcommand{\makeheading}[1]%
        {\hspace*{-\marginparsep minus \marginparwidth}%
         \begin{minipage}[t]{\textwidth+\marginparwidth+\marginparsep}%
                {\large \bfseries #1}\\[-0.15\baselineskip]%
                 \rule{\columnwidth}{1pt}%
         \end{minipage}}

% The section headings
%
% Usage: \section{section name}
%
% Follow this section IMMEDIATELY with the first line of the section
% text. Do not put whitespace in between. That is, do this:
%
%       \section{My Information}
%       Here is my information.
%
% and NOT this:
%
%       \section{My Information}
%
%       Here is my information.
%
% Otherwise the top of the section header will not line up with the top
% of the section. Of course, using a single comment character (%) on
% empty lines allows for the function of the first example with the
% readability of the second example.
\renewcommand{\section}[2]%
        {\pagebreak[2]\vspace{0.7\baselineskip}%
         \phantomsection\addcontentsline{toc}{section}{#1}%
         \hspace{0in}%
         \marginpar{
         \raggedright \scshape #1}#2}

% An itemize-style list with lots of space between items
\newenvironment{outerlist}[1][\enskip\textbullet]%
        {\begin{enumerate}[#1]}{\end{enumerate}%
         \vspace{-1\baselineskip}}

% An environment IDENTICAL to outerlist that has better pre-list spacing
% when used as the first thing in a \section
\newenvironment{lonelist}[1][\enskip\textbullet]%
        {\vspace{-\baselineskip}\begin{list}{#1}{%
        \setlength{\partopsep}{0pt}%
        \setlength{\topsep}{0pt}}}
        {\end{list}\vspace{-.6\baselineskip}}

% An itemize-style list with little space between items
\newenvironment{innerlist}[1][\enskip\textbullet]%
        {\begin{compactenum}[#1]}{\end{compactenum}}

% To add some paragraph space between lines.
% This also tells LaTeX to preferably break a page on one of these gaps
% if there is a needed pagebreak nearby.
\newcommand{\blankline}{\quad\pagebreak[2]}

\usepackage{fontawesome}

%%%%%%%%%%%%%%%%%%%%%%%% End Helper Commands %%%%%%%%%%%%%%%%%%%%%%%%%%%

%%%%%%%%%%%%%%%%%%%%%%%%% Begin CV Document %%%%%%%%%%%%%%%%%%%%%%%%%%%%

\begin{document}

\makeheading{\huge \textsc{Bidhov Bizar} \\
\rule{\textwidth}{1pt}
\large Software Engineer | Cisco | IISc | RIT}

\section{Contact Information}
% NOTE: Mind where the & separators and \\ breaks are in the following
%       table.
%
% ALSO: \rcollength is the width of the right column of the table
%       (adjust it to your liking; default is 1.85in).
\newlength{\rcollength}\setlength{\rcollength}{2.5in}%
%
\begin{tabular}[t]{@{}p{\textwidth-\rcollength}p{\rcollength}}

  %EECS Department\\
Raycon Lotus Apartment       & +91 7012-556-335 \\
AECS Layout, Kundalahalli   &                    bidhovbizar.office@gmail.com   \\
Bangalore, KA 560037, India          & \href{https://ece.iisc.ac.in/~bidhovbizar/}{https://ece.iisc.ac.in/$\sim$bidhovbizar/}\\

\end{tabular}

%\section{Current  Position}
%\textbf{University of California, Berkeley}
%\begin{outerlist}
%\item[] \textit{Postdoctoral Scholar}%
%        \hfill  September 2009 - Present\\
%        Supervisor: Prof. Kannan Ramchandran     
%         \begin{innerlist}
%         
%           \item[$\diamond$] Fundamental limits and codes  for data reliability and security    in  distributed  storage systems
%\item[$\diamond$] Algorithms and codes for a  distributed cache-based Video-on-Demand (VoD)  system that is being built at Berkeley
%% \item[$\diamond$]Researching fundamental limits on data reliability and security    in  distributed  storage systems
%%\item[$\diamond$] Developing efficient algorithms and coding schemes achieving these fundamental bounds
%%\item[$\diamond$]Studying reliable  and secure  data  storage in distributed cloud systems
%%\item[$\diamond$] Research on distributed systems for data storage, energy efficient wireless communications
%%\item [$\diamond$] Supervisor: Prof. Kannan Ramchandran
%\end{innerlist}
%
%%\item[-] \textit{Grader}%
% %       \hfill \textbf{Fall 2004}
%%\begin{innerlist}
%%\item[$\diamond$]Graded homework papers for the \textit{Information Theory} class
%% \end{innerlist}
%\end{outerlist}
%


\section{Professional Interests}
 Distributed Systems and Networking, L2/L3 Network Protocols, Network Telemetry, Switching Technologies, Software-Defined Networking, Network Performance Optimization, Machine Learning

\section{Industrial Experience}
\textbf{Software Engineer at  Data Center BU, Cisco:}
\hfill July,2022-July,2023

\textbf{L3 Team}
\begin{compactitem}
\item Completed CCNA training and scored 29/30.
\item Worked on few features requiring the creation of new CLI commands based on customer requirements from scratch.
\item Worked through 3 release cycles, including responsibilities in SPAN and L3.
\item Configured and debugged ISIS, BFD, VRF, ARP and other L2/L3 network issues.
\item Gained expertise in inter-process communication(IPC) between network components especially between supervisor and line cards.
\item Represented the L3 team for few customer bug and led successful debugging sessions.
\item Collaborated with other component teams to resolve several test stopper bugs on time.
\item Conducted webinars on 'Classification of L3 commands' and 'How to construct a new command and DMEize' for internal knowledge sharing.
\item Played a key role in resolving various L3 consistency checker bugs.
\item Led the resolution of multiple bugs in total, including root cause analysis.
\item Assisted in setting up Nexus Switches in a fat-tree topology and configuring them for testing purposes.
\item Actively contributed to documentation, locally and on the company's wiki.
\end{compactitem}

\textbf{L2 Team}
\hfill Jan,2022-July2022
\begin{compactitem}

\item    Wrote L2 SPAN sanity testsuite for the upcoming Cisco ASIC in a virtual platform.
\item    Utilized Nexus test automation docker to test and solve testsuite issues.
\item    Successfully closed bugs related to SPAN and L2 forwarding.
\item    Gained in-depth knowledge of packet walk in N7k and N9k switch.
\item    Proficiently worked with SPAN and ERSPAN for traffic monitoring.
\end{compactitem}

\textbf{Intern at Cisco:} 
\hfill May,2021-July,2021
\begin{compactitem}
\item Assisted L2 and L3 teams of Data Center BU in diagnosing and resolving component time convergence issues.
\item Aided in identifying the root causes behind the severe increase in delays during the switch's transition from standby to active state.
\item Developed a Python script to automate the identification of convergence issues by comparing timestamps of various components, reduceing the time for  diagnosing similar incidents and improving overall network performance.
\end{compactitem}

\textbf{ Hardware Engineer at TIES:}
\hfill Aug,2015-April,2016
\begin{compactitem}
\item Collaborated with a cross-functional team to develop an automated Bio Gas plant for research in bacteria at Tropical Institute of Ecological Science.
\end{compactitem}
\textbf{Intern at Robosapiens Technologies Pvt. Ltd:} 
\hfill May,2015
\begin{compactitem}
\item Detailed study and analysis of ATMEGA328,ATMEGA168. 
\item Developed a robotic arm control system using ATMEGA328, enabling movement in two axes.  
\end{compactitem}
\vspace{-6pt}
\section{Projects Experience}
\vspace{-.6cm}
\begin{outerlist}
\item {\bf Networking:}\\
Constructed network using {\bf Mininet} with multiple {\bf TRex Traffic Generator} at hosts running stateless, stateful and Client Server L7 traffic and obtained telemetry.[2021]\\
Constructed {\bf \href{https://github.com/bidhovbizar/DCN-Simulation/tree/master/PythonCode}{Simulation Framework}} in {\bf Python} for large scale Data Center in Fat-Tree Topology. [2020]  \\ 
Simulated Fat-Tree topology to observe the the congestion of flow in {\bf NS3}.[2020]\\
Emulated Fat-Tree topology in Mininet to collect the telemetry data using {\bf Ryu} Controller and {\bf OpenVSwitches}.[2020]
\vspace{-6pt}
\item {\bf Java Development:} \\
Deployed {\bf Kafka} system in Network Lab, used Confluent Message Generator to run experiments and optimized it for better performance than off the shelf.[2019]
\vspace{-6pt}
\item {\bf Machine Learning:}\\
Processed the {\bf telemetry} data by applying sketching algorithms and observe the congestion.[2020]\\
 Studied {\bf Multi-label regression} techniques such as xgboost, random forest, Naive Bayes over {\bf sketched} data.[2019]\\
{\bf Binary classification } and predict the trump cards for a game of cards.[2019]
\vspace{-6pt}
\item {\bf Hardware Project:}\\
Developed {\bf Heterogenous Traffic Analyzer} to be deployed in roads to calculate traffic intensity.[Funded By TEQUIP,2016]\\
Created {\bf Electronic Braille reader} to read from text paired with mobile via bluetooth.[Funded by AIYEHUM,IEEE,2015]
\vspace{-6pt}
\item {\bf C++ with SQL:} \\
Created billing system for storage and book keeping purpose, where the database source was stored at MySQL database. [2012]
\end{outerlist}

\section{Computer Proficiency}
\begin{tabular}{l l}
\textbf{Programming Languages:} & Java, Python, C, C++\\
\textbf{Programming Frameworks:} & Embedded C, AVR, Ryu, Pox, NS3 \\
\textbf{Tools \& Technologies:} & Kafka, WireShark, Mininet, TRex, IXIA, Cisco Switches\\
\textbf{Operating System:} & {\sc Linux}, Windows, CentOS, RHEL, SONiC, NX-OS\\
\textbf{Linux Package:} & {\LaTeX}, Dockers\\
\textbf{Scripting:} & {\sc Matlab}, Bash\\
\textbf{Machine Learning Framework:} & Keras, Scikit-learn\\
\end{tabular}

\section{Education}
%
%\begin{outerlist}
%\item[]
%\textbf{Ph.D., Texas A\&M University, College Station} %
%\hfill 07/2004-12/2011\\
%Department of Electrical and Computer Engineering 
%        \begin{innerlist}
%        \item[$\diamond$] Dissertation: \emph{``Delay-sensitive communications: code-Rates, strategies, and distributed control''}
%        \item[$\diamond$] Advisors:     {Jean-Fran\c{c}ois Chamberland \& Srinivas Shakkottai}
%        \item[$\diamond$] Co-Advisor:  {Prof. Alex Sprintson}
%        \end{innerlist}
%\end{outerlist}\vspace{-.1cm}
%\blankline \\
\textbf{M.~Tech.(Research), Indian Institute of Science}% 
\hfill 07/2018-Present\\ %\vspace{-.2cm}
Electrical Communication Engineering, Network Lab
        \begin{innerlist}
        \item[$\diamond$] Dissertation: \emph{``Empirical study of load generation for Data Center Network''}
        \item[$\diamond$] Advisor:     {\href{https://ece.iisc.ac.in/~parimal/}{Parimal Parag}}
        
        \end{innerlist}
\vspace{2pt}

\textbf{B.~Tech., Rajiv Gandhi I.T., Kottayam}%
	\hfill 07/2012-07/2016\\ %\vspace{-.2cm}
Electrical Engineering\\
Class Rank : 7/72
\vspace{2pt}

\textbf{Higher Secondary School, Kendriya Vidyalaya, KGQ}%
	\hfill 06/2010-07/2012\\ %\vspace{-.2cm}
Science Stream \\
Percentage : 94
\vspace{2pt}

\textbf{High School,Kendriya Vidayalaya, KGQ}%
	\hfill 06/2004-07/2010\\
Percentage : 92

\section{Relevant Courses}
\textbf{Electrical Engineering}: 
 Digital Communications, Communication and Sensor Networks, Real Time System, Image Processing.\\
\textbf{Mathematics}: 
Probability, Convex Optimization, Queuing Thoery.\\
\textbf{Data Science}: 
 Machine Learning, Practical Data Science, Pattern Recognition and Neural Network

%\section{Academic Experience}
%\textbf{Research Assistant, CUSAT}%
%        \hfill  06/2015\\
%Department of Physics

%\textbf{Summer Intern, Los Alamos National Lab}%
%\hfill  05/2007 - 08/2007\\
%Computer, Computational and Statistical Sciences\\

\section{Honors \& Awards}
\vspace{-.6cm}
\begin{compactitem}
\item All India rank 1276 in ECE stream in GATE, 2018.
\item Project shortlisted for AIYEHUM,IEEE R10.[2016]
\item Shortlisted for Mathematics camp For INMO (2011,2012)
\item $99^{th}$ rank in {Junior Mathematics Olympiad}, 2010.
\end{compactitem}
\renewcommand{\labelenumi}{\arabic{enumi}.}

%\section{ Publications}
%%
%\textbf{Book Chapters and Theses}\vspace{2mm}
%\begin{enumerate}
%\item [{\bf [B1]}] P.~Parag,.
%\newblock {\em Delay-sensitive communications: code-Rates, strategies, and distributed control.}
%\newblock PhD thesis, Texas A\&M University, College Station, TX, USA, 
%	July 2011. 
%\end{enumerate}
%%
%
%\textbf{Peer-Reviewed Journal Papers}\vspace{2mm}
%
%\blankline
%$\quad${\it In Preparation}
%\begin{enumerate}
%\addtocounter{enumi}{6}
%	
%\item[{\bf [J12]}] S.~C.~Bobbili, P.~Parag and J.-F.~Chamberland.
%\newblock Timely status updates over erasure channels.
%%\newblock {\em IEEE Transactions on Information Theory}, 53(5):1767--1777, 
%	to be submitted.


%\section{Invited Talks}
%
%``Real-Time Status Updates for Markov Sources,''
%\begin{innerlist}
%\item[] \textit{Bombay Information Theory Seminar,},  Jan  2018.
%\item[] \textit{IIT Delhi},  Jan  2017.
% \end{innerlist}
% 
%  \blankline
%  
%``Job completion times in coded parallel systems,''
%\begin{innerlist}
%\item[] \textit{Faculty colloquium at IISc Bangalore},  Jan  2018.
%\item[] \textit{MVJ College of Engineering at Bangalore},  Aug  2017.
%\item[] \textit{Georgia Institute of Technology at Atlanta},  May  2017.
%\item[] \textit{Lectures in Probabilities Seminar XI at ISI Delhi},  Nov 2016.\\
% \end{innerlist}
%  \blankline

%%\section{Posters}
%%``Securing Distributed Storage Systems Against Adversarial Attacks,''
%%\begin{innerlist}
%%\item[]  \textit{DTRA Basic Research Technical Review }, Springfield,  August  2010.
%%\item[]  \textit{School of Information Theory}, Los Angeles,  August  2010.
%%\end{innerlist}
%%
%%\blankline
%%
%%``A New Construction Method for Matroidal Networks,''
%%\begin{innerlist}
%%\item[]  \textit{Winedale Workshop on Signals and Systems}, Round Top, October 2008.
%%\end{innerlist}
%%
%%\blankline
%%
%%``Some Results on the Index Coding Problem,''
%%\begin{innerlist}
%% \item[]   \textit{First Annual School of Information Theory}, State College,   June 2008.
%%\end{innerlist}

%%%%%%%%%%%%%%%%%%%%%%%%%%%%%%%%%%%%%%%%%%%%%%%%%%%%%%%%%%%%
%\newpage
%\section{Teaching Experience}
%\textbf{Indian Institute of Science}
%\begin{outerlist}
%\item[] \textit{Lecturer}%
%               \begin{innerlist}
%                 \item[$\diamond$] E2 204: Stochastic Processes and Queueing Theory \hfill { Spring 2018}
%                 \item[$\diamond$] E2 202: Random Processes \hfill { Fall 2017}
%                 \item[$\diamond$] E1 244: Estimation and Detection Theory \hfill { Spring 2017}
%                 \item[$\diamond$] E2 204: Stochastic Processes and Queueing Theory \hfill { Spring 2017}
%                 \item[$\diamond$] E1 244: Estimation and Detection Theory \hfill { Spring 2016}
%                 \item[$\diamond$] E2 204: Stochastic Processes and Queueing Theory \hfill { Spring 2016}
%                 \item[$\diamond$] E0 201: Proofs and Measures \hfill { Fall 2015}
%                 \item[$\diamond$] E2 204: Stochastic Processes and Queueing Theory \hfill { Spring 2015}
%               \end{innerlist}
%\end{outerlist}
%\blankline  


%\textbf{Rutgers University}\vspace{-.2cm}
%\begin{outerlist}
%\item[] \textit{Lecturer}%
               %\begin{innerlist}
                %\item[$\diamond$] 14:332:312: Discrete Mathematics \hfill { Spring 2013}
               %\end{innerlist}
%\end{outerlist}
%\blankline  

%\textbf{Texas A\&M University}\vspace{-.2cm}
%\begin{outerlist}
%\item[] \textit{Guest Lecturer}%
% \begin{innerlist}
%	\item[$\diamond$] {ECEN 303} -- Random Signals and Systems \hfill { Spring 2010}
%	\item[$\diamond$] {ECEN 601} -- Linear Network Analysis \hfill { Fall 2009}
%	\item[$\diamond$] {ECEN 662} -- Estimation and Detection Theory \hfill { Spring 2009}
%	\item[$\diamond$] {ECEN 683} -- Wireless Communications \hfill { Fall 2008}
%\end{innerlist}
%\end{outerlist}
%\blankline        
   
%%%%%%%%%%%%%%%%%%%%%%%%%%%%%%%%%%%%%%%%%%%%%%%%%%%%%%%


%\section{Graduate Course Work}
%Wireless Communication, Modulation Theory,  Detection and Estimation, Information Theory, Channel Coding, Advanced Digital Signal Processing, Probability Theory and Stochastic Processes,  Queuing Theory,  Number Theory, Algebraic Geometry, Algorithmic Algebraic Geometry, Topology, Algebraic Topology, Elliptic Curves and Modular Forms, Fuzzy Logic, Sustainable Development Management.

%\section{Supervised Students}

%\section {Research Grants}

%\section {Professional Service}
%%\vspace{-.2cm}
%\textbf{Program Chair}
%\begin{innerlist}
%\item [$\diamond$] The ACM International Symposium on Mobile Ad Hoc Networking and Computing (MobiHoc) : Posters and Demonstrations 2018, Workshops 2017
%\item [$\diamond$] International Conference on Signal Processing and Communications (SPCOM): Publications 2018, Web 2016 
%\item[$\diamond$] International Conference on Communication Systems and Networks (COMSNETS): Intelligent Transportation Systems Workshop 2018
%\item[$\diamond$] IEEE International Conference on Advanced Networks and Telecommunications Systems (ANTS 2016): Technical Program Committee.
%
%\end{innerlist}
%\blankline


%\vspace{-.2cm}\textbf{Session Chair}
%\begin{innerlist}
%\item[] Network Coding Workshop (2009), International Symposium on Information Theory (2010).
% \end{innerlist}
% 
%\blankline

%\section{Languages}  English, Arabic, French\\

\section{Memberships}
IEEE, Computer Society, WIE
\blankline

\section{References}\vspace{-5mm}
\begin{center}
\begin{tabular}{ll}
\textbf{Prateek Sadhukhan}  \\
Leader, Software Engineering (Last Manager)\\
Data Center BU\\
Cisco Systems India Pvt Ltd, Bengaluru\\
Tel: (+91)  98808 55911 \\
E-mail: psadhukh@cisco.com\\[1cm]
\end{tabular}
\begin{tabular}{ll}
\textbf{G. Shankar Gopalkrishnan} \\
Principal Software Engineer \\
Data Center BU\\
Cisco Systems India Pvt Ltd, Bengaluru\\
Tel: (+91)  99805 41523\\
E-mail: gshankar@cisco.com\\[1cm]
\end{tabular}

\end{center}

\end{document}

%%%%%%%%%%%%%%%%%%%%%%%%%% End CV Document %%%%%%%%%%%%%%%%%%%%%%%%%%%%%
